\documentclass[openany,9pt,italian,ifnoinputencoding]{dictionaryCD}

\newcommand{\provauno}[2][]{\ifblank{#1}{%
	#2\label{lm@#2}}{#2\label{lm@#1}}}
\begin{document}
	\chapter{A}
	\vspace{20mm} 
	\lemma{prova}
\sublemma{test}

\sublemma[casa]{casè}
\entry[acondrite]{acondriti}
	
	
\vedilemma[casa]{casè}
	
	\vedilemma{prova}
	
	\vedilemma{test}
	
\begin{compactitem}
	\item[\textit{angriti}] di cui esiste un solo esemplare composto dall'augite, un silicato;
	\item[\textit{aubriti}] simili alle condriti con un basso tasso ferroso: si ipotizza che la loro origine provenga dalla più interna fascia asteroidale;
	\item[\textit{ureiliti}] rare meteoriti con granuli con modesta presenza di carbonio, circa il 2~\%;
	\item[\textit{HED}] sigla che deriva dalle iniziali di tre gruppi  di meteoriti cui la subfamiglia fa riferimento: \textit{Howarditi}, \textit{Eucriti}, \textit{Diogeniti}. Le Eucriti hanno composizione simile ai basalti terrestri, le Diogeniti ai cumulati pirossenici, le Howarditi da più gruppi petrosi. Per questo gruppo metereoidale si suppone una origine \textit{spaziale} coincidente con \seeentry{Vesta}, il terzo asteroide per dimensioni, le cui osservazioni prima telescopiche e poi spaziali hanno evidenziato una decisa relazione fra questo tipo di acondriti e l'asteroide;
	\item[\textit{SNC}] sigla che deriva dalle iniziali dei gruppi di meteoriti cui la subfamiglia fa riferimento: \textit{Shergottiti}, \textit{Nakhliti}, \textit{Chassigniti}, nomi che derivano dalle località in cui sono stati trovati: Shergathi in India, Nakhla in Egitto e Chassigny in Francia. Si suppone siano il frutto di qualche eruzione vulcanica in epoca non eccessivamente remota, non oltre 1,4~miliardi di anni, questa l'età massima della loro cristallizzazione, ed alcune di queste non superano i 180~milioni di anni, indizi che propendono per un'origine marziana la cui attività vulcanica era presente a quelle età. Appartiene a queste la  meteorite \seeentry{ALH84001}.
\end{compactitem}	

\begin{table}[t!]%
	\begin{center}%
		\begin{small}\tabcolsep=2.6pt%
			\begin{shadedWtabular}[celeste]{*6l}%
				\toprule%
				\multicolumn6c{crux ed Acubens, dati e caratteristiche all'equinozio 2000}\\%
				\midrule%
				\multicolumn6c{Acrux}\\%
				$\varphi$ & 12\hour\ 26\minute\ 35,89\second   & $\delta$ &-63\gradi\ 05' 56,73'' & $M\ped{ap} 0,77 $       &$ M\ped{as} -4,14$\\%
				C\ped{s}    & B1V                                  &                          &        & B-V  & - 0,24   \\
				$\pi$     & $10,17 \pm\ 0,67$\ped{M} & $ly$ &$320 \pm\ 20$ & M       &$ 2$\\%
				%\midrule%
				\multicolumn6l{Design.: HR 4730, HD~108248, SAO~251904, FK5~462, HIP~60718 }\\%
				\multicolumn6l{}\\%
				\multicolumn6c{Acubens}\\
				$\varphi$ & 08\hour\ 58\minute\ 29,22\second   & $\delta$ &+11\gradi\ 51' 27,72'' & $M\ped{ap} 4,20 $       &$ M\ped{as} 0,6$\\
				C\ped{s}    & A5m                                  & U - B                         &   0,15     & B-V  &  0,14\\%
				$\pi$     & $18,79 \pm\ 0,99$\ped{M} & $ly$ &$174 \pm\ 9$ & M       &$ 14 \pm\ 4$\\
				\multicolumn6l{Design.: HR 3572, HD~76576, SAO~98267, FK5~337, HIP~44066}\\%
				\bottomrule%
\end{shadedWtabular}\end{small}\end{center}\end{table}%	
	
\begin{minitoc}
	\mttitle{Le galassie di Seyfert}
	\mttitle{Le Radiogalassie}
	\mttitle{I Quasars}
	\mttitle{Oggetti BL Lacertae}
	\mttitle{Modello standard e unificazione degli AGN}
\end{minitoc}	
	
	
\end{document}
