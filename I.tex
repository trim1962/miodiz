% !TeX encoding = UTF-8
% !TeX spellcheck = it_IT
% !TeX root = MatDiz.tex
\chapter{I}
\vspace{5mm} 
\lemma{incognita}Quantità non nota, non conosciuta. L'uso delle incognite venne introdotto dal matematico francese Fraçois Viète\pointsto~\seeentry[Viete Francois]{Viète François} nel 1571.\index{Viète Fraçois}
\lemma{insieme}Lista, collezione o classe di oggetti
che sia ben definita o per meglio dire univocamente definita. Per indicare un insieme si usano le lettere maiuscole. Per indicare un elemento si usano le lettere minuscole. Un insieme può essere definito per elencazione o tramite la proprietà che lo caratterizza.\cite{Lipschutz1980}
\lemma{intervallo}Sottoinisieme della retta reale compreso tra due estremi detti estremo sinistro e destro\pointsto~\vedilemma{segmento}.\llemma{i. illimitato superiormente} ha l'estremo destro è $+\infty$. \llemma{i. illimitato inferiormente} ha l'estremo sinistro è $-\infty$.\llemma{i. illimitato} ha per estremo sinistro $-\infty$ e per estremo destro $+\infty$.
\lemma{integrale}
\lemma{ipotenusa}Lato maggiore di un triangolo rettangolo\pointsto~\vedilemma{t. rettangolo}.

