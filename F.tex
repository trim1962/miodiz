% !TeX encoding = UTF-8
% !TeX spellcheck = it_IT
% !TeX root = MatDiz.tex
\chapter{F}
\vspace{5mm}
\lemma{fattoriale}
\lemma{figura}Ogni insieme di punti o linee o  superfici.\llemma{f. piana}Figura tutta contenuta in un piano.
\entry{funzione}
Una funzione è una relazione\pointsto~\seeentry{relazione} che ad ogni elemento di $x\in A$ dominio\pointsto~\seeentry{dominio}, fa corrispondere uno e uno solo elemento appartenente a $B$ codominio.
\begin{equation*}
\function{f}{A}{B}{x}{f(x)}
\end{equation*}
\llemma{f. algebrica}
Una funzione $\funzione{f}{A}{B}$ si dice algebrica\pointsto~\vedilemma{algebrico} se costruita utilizzando un numero finito di applicazioni delle quattro operazioni dell'aritmetica, dell'elevazione a potenza e delle radici.
\llemma{f. algebrica razionale}
Una funzione $\funzione{f}{A}{B}$ algebrica è razionale quando la variabile indipendente non si trova sotto il segno di radice.\llemma{f. algebrica irrazionale}Una funzione $\funzione{f}{A}{B}$ algebrica è irrazionale quando la variabile indipendente si trova sotto il segno di radice.
\llemma{f. algebrica intera}Una funzione $\funzione{f}{A}{B}$ algebrica è intera quando la variabile indipendente non si trova al denominatore di una frazione.\llemma{f. algebrica fratta} una funzione $\funzione{f}{A}{B}$ algebrica\pointsto~\vedilemma{algebrico} è fratta quando la variabile indipendente si trova al denominatore di una frazione.\llemma{f. trascendente}Una funzione $\funzione{f}{A}{B}$ è trascendente quando compaiono operazioni non algebriche come logaritmo, esponenziale, goniometriche.
\llemma{f. biettiva}Una funzione $\funzione{f}{A}{B}$ è biettiva se è contemporaneamente suriettiva e iniettiva.
\llemma{f. crescente}Una funzione $\funzione{f}{A}{B}$ è crescente in senso stretto nell'intervallo $I\subset A$ se
$\forall\; x_1,x_2\in I\quad x_1< x_2\Longrightarrow f(x_1)<f(x_2).$
%%%%%%%%
\llemma{f. dispari}Una funzione $\funzione{f}{A}{B}$ è una funzione dispari se $f(-x)=-f(x)\quad\forall x\in A$.\llemma{f. decrescente}Una funzione $\funzione{f}{A}{B}$ si dice decrescente in senso stretto nell'intervallo $I\subset A$ se
$\forall\; x_1,x_2\in I\quad x_1< x_2\Longrightarrow f(x_1)>f(x_2)$.\llemma{f. iniettiva}Una funzione $\funzione{f}{A}{B}$ è iniettiva se
$x_1\neq x_2\quad\Longrightarrow\quad f(x_1)\neq f(x_2)\quad \forall x_1,x_2\in A $ oppure $f(x_1)= f(x_2)\quad\Longrightarrow\quad x_1= x_2\quad \forall x_1,x_2\in A$.\llemma{f. non crescente}Una funzione $\funzione{f}{A}{B}$ si dice non crescente nell'intervallo $I\subset A$ se $\forall\; x_1,x_2\in I\quad x_1< x_2\Longrightarrow f(x_1)\geq f(x_2)$.\llemma{f. non decrescente}Una funzione $\funzione{f}{A}{B}$ si dice non decrescente nell'intervallo $I\subset A$ se $\forall\; x_1,x_2\in I\quad x_1< x_2\Longrightarrow f(x_1)\leq f(x_2)$. \llemma{f. limitata}Una funzione $\funzione{f}{A}{B}$ è una funzione limitata se $\exists\; M\quad \abs{f(x)}<M \quad\forall x\in A$.\llemma{f. periodica}\'{E} una funzione $\funzione{f}{A}{B}$ di periodo $T>0$ se $f(x+kT)=f(x)\quad k\in \Z$\llemma{f. pari} una funzione $\funzione{f}{A}{B}$ è una funzione pari se $f(-x)=f(x)\quad\forall x\in A$.\llemma{f. suriettiva}Una funzione $\funzione{f}{A}{B}$ è suriettiva se $f(A)=B$\llemma{f. zeri}\pointsto~\vedilemma{z. funzione}
\lemma{frazione}
\llemma{f. egizia} 
Una frazione è egizia quando viene scritta come somma di frazioni con numeratore unitario.\cite{Boyer1980}. Metodi di calcolo ed esempi sono stati trovati nel papiro di Rhind\pointsto~\vedilemma{Papiro di Rhind}.
