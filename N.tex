% !TeX encoding = UTF-8
% !TeX spellcheck = it_IT
% !TeX root = MatDiz.tex
\chapter{N}
\vspace{5mm}
\lemma{NAND}Operazione logica. Tabella di verità associata è~\vref{tab:NfunzioneNAND}
\begin{table}
	\scaptiona{Funzione NAND}
	\label{tab:NfunzioneNAND}
	\centering
	\begin{tabular}{ccc}
		\toprule
	$A$&$B$&$A\mathbin{\overline{\wedge}}B$\\
	\midrule
	0&0&1\\
	1&0&1\\
	0&1&1\\
	1&1&0\\
	\bottomrule
	\end{tabular}
\end{table}
\lemma{Napier}\index{Napier} Definì i logaritmi inventandone il termine.
\lemma{Newton Isaac}\index{Newton Isaac}
\begin{figure}
	\centering
	\scaptionb{Isaac Newton (1642-1727)}
	\label{fig:godfreykneller-isaacnewton-1689}
	\includegraphics[width=0.7\linewidth]{Figure/N/GodfreyKneller-IsaacNewton-1689}
\end{figure}
\lemma{NOT}Operazione logica. Tabella di verità associata è~\vref{tab:NfunzioneNOT}
\begin{table}
	\centering
	\begin{tabular}{cc}%
		\toprule
		$A$&$\overline{A}$\\
		\midrule 
		1&0\\
		0&1\\
		\bottomrule
	\end{tabular}
	\scaptiona{Funzione NOT}
\label{tab:NfunzioneNOT}
\end{table}
\lemma{numero}
\llemma{n. complesso} \llemma{n. irrazionale}Un numero è irrazionale se non è possibile scriverlo come rapporto fra due numeri.\llemma{n. razionale} Un numero è razionale se può essere scritto come rapporto di due numeri interi. Il secondo numero deve essere diverso da zero $q=\frac{a}{b}\quad a,b\in \Z\; b\neq 0$. \llemma{n. trascendente}. \llemma{n. transfinito}
\begin{figure}%[!htb]
	\def\FrameCommand{\fboxsep=\FrameSep \colorbox{shadecolor}}% da def\FrameCommand{\fboxsep=\FrameSep\colorbox{shadecolor}}% da "shaded"
	\begin{MakeFramed}{\advance\hsize-\width \FrameRestore}%    da "framed"
		\begin{center}%
			\textcolor{StrongGray}{\textsf{$\sqrt{2}$ è irrazionale}}%
			\par%
			\vspace*{-\smallskipamount}%
			\vrule height 0.8pt width 56mm%
		\end{center}%
		\begin{small}%
			Supponiamo che $\sqrt{2}$ è un numero razionale. Se è razionale allora:
			\begin{align*}
		\sqrt{2}=&\dfrac{a}{b}
		\intertext{con $a$ e $b$ due numeri irriducibili.  Elevando al quadrato}
		2=&\dfrac{a^2}{b^2}\\
a^2=&2b^2
\intertext{quindi se $a^2$ è un multiplo di due $a$ è pari. Ma se $a$ è pari}
a=&2k\\
4k^2=&2b^2\\
b^2=&4k^2\\
\intertext{quindi se $b^2$ è un multiplo di due $b$ è pari}
			\end{align*}
Quindi abbiamo due numeri irriducibili che abbiamo dimostrato essere entrambi pari e quindi riducibili. Assurdo $\sqrt{2}$ non può essere razionale.			
		\end{small}%
		\vspace*{-\smallskipamount}%
	\end{MakeFramed}%
\end{figure}%
\lemma{numerabile}Un insieme è numerabile se può essere messo in corrispondenza con l'insieme dei numeri naturali.