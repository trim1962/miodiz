% !TeX encoding = UTF-8
% !TeX spellcheck = it_IT
% !TeX root = MatDiz.tex
\chapter{S}
\vspace{5mm} 
\lemma{segmento}Parte di una retta compresa tra due punti distinti della retta.\llemma{s. adiacente}Due segmenti sono adiacenti se appartengono alla stessa retta.\llemma{s. consecutivo}Due segmenti sono consecutivi se hanno un vertice\pointsto~\seeentry{vertice} in comune.
\lemma{semiperimetro}La metà del perimetro\pointsto~\seeentry{perimetro}. Viene indicato con il simbolo $P$.
\entry{semipiano}Ciascuna delle due parti in cui viene divisa un piano\pointsto~\seeentry{piano} da una retta.
\entry{semiretta}Ciascuna delle due parti in cui viene divisa una retta\pointsto~\seeentry{retta} da un punto.
\lemma{seno}L'origine del termine è la seguente. In sanscrito il termine per indicare seno era \textit{jya-ardha} semi corda, abbreviato in \textit{jya}. Gli arabi tradussero questo termine in \textit{jiba}. Tale parola era scritta \textit{jyb}. I primi traduttori latini scambiarono questo vocabolo con \textit{jaib} che significa baia e tradussero tale parola con \textit{sinus} che ha analogo significato \cite{Gheverghese2000}.\titolettoa{Definizione} dato un triangolo rettangolo\pointsto~\vedilemma{t. rettangolo} definiamo seno dell'angolo il rapporto tra il cateto\pointsto~\vedilemma{cateto} opposto\pointsto~\vedilemma{o. lato} all'angolo e l'ipotenusa\pointsto~\vedilemma{ipotenusa} $\sin\beta=\frac{\text{opposto}}{\text{ipotenusa}}$. \titolettoa{Definizione} data una circonferenza goniometrica\pointsto~\vedilemma{c. goniometrica} diremo seno l'ordinata\pointsto~\vedilemma{ordinata} del punto di intersezione tra il raggio\pointsto~\vedilemma{raggio} e la circonferenza \vref{fig:sdefinizioneseno}.
\titolettoa{Definizione} data una circonferenza il seno è la meta di una corda\pointsto~\vedilemma{corda} che unisce due punti distinti sulla circonferenza. \llemma{f. seno}
\begin{figure}
	\scaptionb{Funzione seno grafico}
	\label{fig:sfunzioneSeno}
	\includestandalone[width=0.8\linewidth]{Figure/senografico}
\end{figure}
La funzione seno\pointsto~\vedilemma{goniometria} è definita in $\R$  a valore in  $[-1,1]$. La funzione  è limitata\pointsto~\vedilemma{f. limitata}.
La funzione è periodica\pointsto~\seeentry{f. periodica}  di periodo $k\pi\quad k\in\Z$. La figura~\vref{fig:sfunzioneSeno} rappresenta il grafico della funzione.
\begin{figure}
	\scaptionb{Seno definizione}
	\label{fig:sdefinizioneseno}
	\includestandalone[width=\linewidth]{Figure/senodefinizione}
\end{figure}
\lemma{SI}Sistema Internazionale delle unità di misura.
\lemma{sistema di riferimento}sistema che permette di individuare la posizione di un punto.\llemma{s. di riferimento cartesiano} 
\lemma{sessagesimale}Sistema di numerazione a base sessanta. Di origine babilonese viene usato per il calcolo del tempo e nella misura degli angoli.
\entry{somma}Risultato dell'addizione\pointsto~\seeentry{addizione}.