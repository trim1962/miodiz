% !TeX encoding = UTF-8
% !TeX spellcheck = it_IT
% !TeX root = MatDiz.tex
\chapter{O}
\vspace{5mm}
\lemma{OR}Operazione logica. La tabella di verità associata è~\vref{tab:OfunzioneOR}
\begin{table}
	\scaptiona{Funzione OR}
	\label{tab:OfunzioneOR}
	\centering
	\begin{tabular}{ccc}
	\toprule
$A$&$B$&$A+B$\\
\midrule         
0&0&0\\
1&0&1\\
0&1&1\\
1&1&1\\
\bottomrule
\end{tabular}
\end{table}
\lemma{ordinata}
\lemma{opposto}Ha significati sia geometrici che numerici.\llemma{o. angolo}Due angoli sono opposti se hanno il vertice in comune e i lati di uno sono  il prolugamento dell'altro.
\llemma{o. numero}In matematica dato un numero $a$ il suo opposto è quel numero che sommato ad $a$ da come somma\pointsto~\vedilemma{somma} zero\pointsto~\vedilemma{zero}.\llemma{o. vertice}Un vertice\pointsto~\vedilemma{vertice} è opposto ad un lato se non è uno dei suoi estremi.\llemma{o. lato} In un triangolo un lato\pointsto~\vedilemma{lato} è opposto ad un angolo\pointsto~\vedilemma{angolo} se non concorre a delimitarlo.